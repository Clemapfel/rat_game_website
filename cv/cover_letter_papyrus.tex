%! Author = clem
%! Date = 06.07.21

% Preamble
\documentclass[11pt]{article}

\usepackage{amsmath}
\usepackage[sfdefault]{roboto}
\usepackage[T1]{fontenc}
\usepackage{array, booktabs}
\usepackage[x11names,table]{xcolor}
\usepackage{textcomp}
\usepackage{graphicx}
\usepackage{hyperref}
\usepackage{float}
\usepackage{tikz}
\usepackage[absolute,overlay]{textpos}
\usepackage{contour}
\usepackage{multicol}
\usepackage{titlesec}
\usepackage{pifont}
\usepackage{ragged2e}
\usepackage[geometry]{ifsym}
\usepackage[pscoord]{eso-pic}% The zero point of the coordinate systemis the lower left corner of the page (the default).
\usepackage{calc}
\usepackage{listings}

% margin

% color
\definecolor{light_grey}{rgb}{0.9, 0.9, 0.9}
\definecolor{muted_grey}{rgb}{0.5, 0.5, 0.5}
\definecolor{my_pink}{rgb}{0.92, 0.11, 0.475}
\definecolor{light_mint}{rgb}{0.117, 0.9, 0.5686}
\definecolor{dark_mint}{rgb}{0, 0.55, 0.1}
\definecolor{my_aqua}{rgb}{0, 1, 0.982}
\definecolor{grey_backdrop}{rgb}{1, 1, 1}

%\titlespacing*{\section}{0pt}{0ex}{0ex}

\newcommand{\timeline}{\color{my_pink}\makebox[0pt]{\large---}\hskip-0.5pt\vrule width 1pt\hspace{\labelsep}}
\newcommand{\muted}{\color{muted_grey}}
\newcommand{\emptyline}{\,\newline}

\newcommand\textline[4][t]{%
  \par\smallskip\noindent\parbox[#1]{0.6\textwidth}{\raggedright#2}%
  \parbox[#1]{0.5\textwidth}{\raggedleft#4}\par\smallskip%
}

% margin
\usepackage[a4paper, total={\paperwidth -160pt, \paperheight}]{geometry}
\addtolength{\topmargin}{0.2\paperheight}

% Document
\begin{document}

% left side
\begin{tikzpicture}[remember picture, overlay]
\node[anchor=north west, yshift=4pt, xshift=-5pt]
    at (current page.north west)
    {\includegraphics[height=\paperheight]{cover_letter_side}};
\end{tikzpicture}

\begin{textblock*}{\paperwidth}(75pt, 40pt)
\flushleft
\:\newline
{\muted \small to}\linebreak
Hendrik Ramps\linebreak
Papyrus Author Team\linebreak
Pettenkoferstra{\ss}e 16-18\linebreak
10247 Berlin\linebreak
\newline
\textbf{Regarding}: \href{https://papyrus.join.com/jobs/2575350-c-c-software-developer-for-creative-writing-suite?pid=ddb0f7f50cdf3fec0579&utm_source=xingBasicXml&utm_medium=free&utm_campaign=xing&utm_content=c%2Bc%2Bsoftware%2Bdeveloper%2Bfor%2Bcreative%2Bwriting%2Bsuite#apply-window}{C/C++ Software Developer for Creative Writing Suite}
\end{textblock*}

\begin{textblock*}{\paperwidth}(-50pt, 40pt)
\flushright
{\muted \small from}\linebreak
{\Large \textbf{Clemens Cords}}\linebreak
12203 Berlin\linebreak
\newline
\href{mailto:mail@clemens-cords.com}{mail@clemens-cords.com}\linebreak
+49 30 239 438 71 \linebreak
\newline
\newline
\end{textblock*}

\begin{flushright}
Berlin, July 18th 2021
\end{flushright}

\begin{flushleft}
Dear Mr. Ramps
\emptyline
\emptyline
I am applying for the position of C/C++ software developer.\linebreak
\emptyline
Reviewing your application, I am not only excited to see that my skill set and work experience
matches your stated job description, but that the position and the field of word processing and
UI design in general aligns strongly with my passions and background.\linebreak
\emptyline
I not only have an excellent foundation in C/C++ and general software-design, both in
procedural and object-oriented environments, but am very familiar with large-volume high
performance text and image processing from my bioinformatics background.
While not seeming immediately related, there are few practical differences between processing a genome
in text form and processing plain text.
The same is true for medical images and illustrations and I am confident that my theoretical background and hands-on experience transfer directly.
\emptyline
\emptyline
I've also worked as an indie video game dev these past 3 years and in the process gained a strong proficiency in designing, implementing and delivering intuitive, well-optimized and aesthetically
pleasing UI and user experience in general.
I have already had experience with QT from the early stages of our project and I'm confident that I will be able to readjust to it in very short time.
Similarly, I tend to adjust quickly to another team's code base {\muted\textasteriskcentered} and workflow.\linebreak
\emptyline
Please take a look at my attached CV to get a better and more detailed picture of me and my skill set and feel free to reach out to me via my contact info at the top of this page or my CV.\linebreak
\emptyline

Kind regards,\newline
\emptyline
\emptyline
\emptyline
\emptyline

\begin{tikzpicture}[remember picture, overlay]
\node[yshift = 20pt, xshift = 65pt]
{\includegraphics[scale=0.3]{/home/clem/Documents/unterschrift.png}};
\end{tikzpicture}
\rule[0pt]{30ex}{0.5pt}\newline
{\muted Clemens Cords}
\end{flushleft}

\begin{flushright}
\emptyline
\emptyline
{\small\color{light_grey}\textasteriskcentered \:even if C++ \lstinline{goto}s are present}
\end{flushright}


\end{document}